% ICLR 2026 submission (V2)
\documentclass{article}
\usepackage{iclr2026_conference,times}
\usepackage{amsmath,amssymb}
\usepackage{graphicx}
\usepackage{booktabs}
\usepackage{xspace}
\usepackage{siunitx}
\usepackage{hyperref}

% --- Custom Commands ---
\newcommand{\wf}{WF\xspace}
\newcommand{\twop}{2P\xspace}
\newcommand{\methodname}{Poisson-aware Diffusion Guidance\xspace}
\newcommand{\methodacr}{PKL-DG\xspace}

% --- Switches ---
\newif\ifuseDDIMfile
\useDDIMfiletrue % set to \useDDIMfiletrue to include tex/DDIM.tex

% --- Document Title ---
\title{Microscopy Denoising Diffusion with Poisson-aware Physical Guidance}

\author{Anonymous Authors\\
Anonymous Institution}

\begin{document}
\maketitle

% --- Abstract ---
\begin{abstract}
Denoising diffusion models for inverse problems often enforce measurement consistency via an L$_2$ penalty, implicitly assuming homoscedastic Gaussian noise. In photon-limited microscopy, noise is fundamentally Poisson, making L$_2$ guidance statistically mismatched. We introduce Poisson-Kullback--Leibler (PKL) guidance, a physically grounded, adaptive update derived from the Poisson likelihood that integrates a measured forward model (PSF convolution and background). Our framework operates on real grayscale microscopy data with measured PSFs and detector backgrounds; no synthetic training is used. We provide a rigorous evaluation against strong baselines, including Anscombe-stabilized L$_2$ and Richardson--Lucy deconvolution, and conduct extensive ablations (guidance type/strength schedules, PSF sources, and model options). Placeholders for quantitative results and figures are included; code will be released at submission to ensure reproducibility.
\end{abstract}

% --- Introduction ---
\section{Introduction}
\label{sec:intro}
Denoising diffusion probabilistic models (DDPMs) have become powerful priors for solving imaging inverse problems~\cite{ho2020ddpm, song2022solving}. A standard approach steers sampling to agree with a physical forward model $\mathcal{A}$ and measurement $\mathbf{y}$ via a data-consistency correction~\cite{chung2022diffusion}. Most works adopt an L$_2$ penalty, which presumes Gaussian noise.

In fluorescence microscopy, photon arrivals follow Poisson statistics. The variance is signal dependent, and Gaussian assumptions can over-penalize bright regions while under-correcting faint, photon-starved structures. This mismatch degrades reconstructions precisely where biological signal is scarce.

We propose \emph{Poisson-Kullback--Leibler (PKL) guidance}, replacing L$_2$ with the KL divergence associated with the Poisson likelihood. PKL yields a simple, intuitive update that scales inversely with expected intensity and incorporates measured background. We evaluate on real grayscale microscopy with measured PSFs and demonstrate improved fidelity and robustness compared to L$_2$ and an Anscombe-stabilized L$_2$ baseline, along with a classical Richardson--Lucy deconvolution.

Our contributions are:
\begin{enumerate}
    \item A physically grounded diffusion guidance mechanism derived from Poisson statistics, with an adaptive schedule for stable sampling.
    \item A unified framework coupling a measured PSF-based forward model and background with diffusion, evaluated on real microscopy data.
    \item Comprehensive ablations of guidance design choices and robustness (PSF mismatch, alignment), with placeholders for results and figures; code to be released at submission.
\end{enumerate}

% --- Methodology ---
\section{Methodology}
\label{sec:methodology}
Our method combines a diffusion model prior with physics-informed guidance built from a measured forward model.

\subsection{Physical Forward Model}
Let $\mathbf{x} \in \mathbb{R}_+^{H\times W}$ denote the clean, high-resolution image (\twop-like), and $\mathbf{y} \in \mathbb{N}^{H\times W}$ the observed widefield (\wf) photon counts. We model acquisition as
\begin{equation}
\mathbf{y} \sim \mathrm{Poisson}(\mathcal{A}(\mathbf{x}) + B),
\end{equation}
where $\mathcal{A}$ is a linear convolution with a \emph{measured} PSF, and $B$ is a \emph{measured} background term (dark counts and out-of-focus/background fluorescence). We retain $\mathbf{y}$ as integer counts and de-normalize the model's prediction before applying the forward model.

\subsection{Diffusion Preliminaries}
Following DDPM, the model predicts noise $\boldsymbol{\epsilon}_\theta(\mathbf{x}_t, t)$ and we form an estimate of the clean image
\begin{equation}
\hat{\mathbf{x}}_0(\mathbf{x}_t,t) = \frac{1}{\sqrt{\bar{\alpha}_t}}\left(\mathbf{x}_t - \sqrt{1-\bar{\alpha}_t}\,\boldsymbol{\epsilon}_\theta(\mathbf{x}_t, t)\right).
\end{equation}
This estimate is corrected using a physics-informed guidance step.

\subsection{Training and Inference Paradigms}
\textbf{Training.} We train an \emph{unconditional} diffusion prior over \twop images with a standard DDPM objective. To gently encourage physics-consistency without sacrificing prior generality across microscopes/PSFs, we optionally add a \emph{light} forward-consistency regularizer: simulate \wf by applying the measured forward model to $\hat{\mathbf{x}}_0$ and penalize discrepancy in the measurement domain. This term is small and warmed up during training.

\textbf{Inference.} At sampling time, \wf enters through physics-informed guidance using the same measured forward model. Our implementation supports multiple guidance types (L$_2$, Anscombe+L$_2$, PKL) under a common schedule and gradient normalization. All experiments use \emph{real microscopy} with measured PSFs and backgrounds.

\subsection{Physics-Informed Guidance}
\paragraph{L$_2$ Guidance (Baseline).}
\begin{equation}
\nabla_{\text{L2}} = \mathcal{A}^T\big(\mathbf{y} - (\mathcal{A}(\hat{\mathbf{x}}_0) + B)\big).
\end{equation}

\paragraph{Anscombe+L$_2$ Guidance (Strong Baseline).}
Apply $f(z)=2\sqrt{z + 3/8}$ before L$_2$:
\begin{equation}
\nabla_{\text{Anscombe}} = \mathcal{A}^T\,\nabla_{\mathbf{x}}\,\lVert f(\mathbf{y}) - f(\mathcal{A}(\hat{\mathbf{x}}_0)+B)\rVert^2.
\end{equation}

\paragraph{Proposed PKL Guidance.}
Using the KL divergence for Poisson,
\begin{equation}
\nabla_{\text{PKL}} = \mathcal{A}^T\left(1 - \frac{\mathbf{y}}{\mathcal{A}(\hat{\mathbf{x}}_0)+B+\epsilon}\right),
\end{equation}
with small $\epsilon$ for numerical stability.

\subsection{Adaptive Guidance Schedule}
We compute a corrected estimate
\begin{equation}
\hat{\mathbf{x}}'_0 = \hat{\mathbf{x}}_0 - \lambda_t\,\nabla,
\end{equation}
with
\begin{equation}
\lambda_t = \frac{\lambda_{\text{base}}}{\lVert\nabla\rVert_2 + \epsilon_\lambda}\cdot \min\left(\frac{T-t}{T-T_{\text{thr}}},\,1.0\right).
\end{equation}
This schedule normalizes step size and ramps guidance after a threshold $T_{\text{thr}}$ for stability.

\subsection{Implementation Details and Design Choices}
To ensure clarity and reproducibility, we enumerate the key design decisions that affect stability and performance:
\begin{itemize}
    \item \textbf{Data normalization:} Training images are normalized to $[-1,1]$; at inference, $\hat{\mathbf{x}}_0$ is de-normalized to non-negative intensities before applying $\mathcal{A}$. We clip to $\ge 0$ post-correction.
    \item \textbf{Background handling:} $B$ is measured per dataset/session; we ablate global scalar vs. per-image estimates. We avoid subtracting $B$ from raw counts to preserve Poisson structure.
    \item \textbf{Stability constants:} $\epsilon$ (Poisson ratio) and $\epsilon_\lambda$ (scheduler) are tuned on validation; we report ranges and sensitivity.
    \item \textbf{Gradient normalization:} We use global L2 normalization as above; we ablate per-pixel vs. global scaling and alternative norms.
    \item \textbf{Guidance ramp:} $T_{\text{thr}}$ controls warm-up; we ablate linear vs. cosine ramps and thresholds.
    \item \textbf{Forward model fidelity:} Measured PSFs are used by default; we ablate against Gaussian PSFs and broadened/mismatched PSFs to assess robustness.
    \item \textbf{Alignment:} We quantify tolerance to small misregistrations by evaluating on imperfectly registered pairs.
    \item \textbf{Diffusion options:} Learned variance and EMA are toggled per configuration; we report their effect. Mixed precision and memory optimizations (channels-last, gradient checkpointing) are used as engineering choices and ablated when relevant.
\end{itemize}
These points address prior “method detail gaps” by specifying normalization, numerical safeguards, and forward-model usage.

% --- Related Work and Positioning ---
\section{Related Work and Positioning}
\label{sec:related}
\textbf{Diffusion for inverse problems.} Numerous works guide diffusion sampling with physics-based constraints or data consistency steps~\cite{song2022solving, chung2022diffusion, kawar2022denoising, saharia2022image, rombach2022high}. Our work falls into the family of likelihood- or gradient-based guidance that couples a forward operator with a generative prior.

\textbf{Poisson noise in imaging.} Classical approaches include Richardson--Lucy deconvolution~\cite{richardson1972bayesian, lucy1974iterative} and variance-stabilizing transforms like Anscombe~\cite{anscombe1948}. Modern restoration often applies Gaussian models post-transform; we compare directly to this stronger baseline.

\textbf{Physics-aware guidance.} Prior diffusion methods have considered data-fidelity terms matched to noise models, but most microscopy applications still default to L$_2$. We explicitly derive and evaluate a Poisson-appropriate KL guidance with measured backgrounds, showing benefits in low-count regimes. Our positioning is complementary to plug-and-play/RED~\cite{venkatakrishnan2013plug, romano2017red} and score-based data consistency, emphasizing principled Poisson modeling for microscopy.

% --- Experimental Design ---
\section{Experimental Design}
\label{sec:experiments}
\subsection{Data and Model Details}
\textbf{Dataset:} Real grayscale widefield (\wf) and two-photon (\twop) microscopy, with measured PSFs (beads) and measured detector backgrounds.

\textbf{Implementation:} Conditional U-Net backbone trained on real microscopy with diffusion objectives; mixed precision on multi-GPU hardware. Inference uses DDIM with $N$ steps. Hyperparameters are selected on validation.

\subsection{Models for Comparison}
We compare to:
\begin{itemize}
    \item \textbf{WF Input (passthrough).}
    \item \textbf{Richardson--Lucy deconvolution (RL)} with measured PSF.
    \item \textbf{L$_2$-guided Diffusion.}
    \item \textbf{Anscombe+L$_2$-guided Diffusion.}
    \item \textbf{PKL-guided Diffusion (ours).}
\end{itemize}
\textit{Note:} We omit RCAN supervision because our data are single-channel grayscale; channel-attention architectures are not directly applicable or competitive under our constraints.

\subsection{Evaluation Protocol \& Results}
\textbf{Reconstruction Fidelity and Resolution.} We report PSNR, SSIM, and Fourier Ring Correlation (FRC).

% [Table 1 Placeholder: Quantitative comparison across methods]
% Columns: Method, PSNR$\uparrow$, SSIM$\uparrow$, FRC (nm)$\downarrow$.

% [Figure 1 Placeholder: Visual comparison with challenging regions]

\textbf{Downstream Scientific Task Performance.} We evaluate neuron segmentation with Cellpose and morphological accuracy via Hausdorff distance.

% [Table 2 Placeholder: Downstream task performance]

\textbf{Robustness to Model Mismatch.} We test PSF broadening and misalignment.

% [Figure 2 Placeholder: Robustness grid for L$_2$, Anscombe+L$_2$, PKL]

% --- Ablation Studies ---
\section{Ablation Studies}
We systematically ablate guidance and system design choices using the unified evaluation harness.
\begin{itemize}
    \item \textbf{Guidance type:} L$_2$ vs. Anscombe+L$_2$ vs. PKL.
    \item \textbf{Guidance schedule:} $\lambda_{\text{base}}$, $T_{\text{thr}}$, normalization choice.
    \item \textbf{Stability constants:} $\epsilon$, $\epsilon_\lambda$.
    \item \textbf{Background modeling:} global vs. per-image $B$; background misestimation sensitivity.
    \item \textbf{PSF source:} measured vs. Gaussian vs. broadened/mismatched.
    \item \textbf{Conditioning:} with/without explicit \wf conditioning channel (if applicable).
    \item \textbf{Sampler steps:} number of DDIM steps.
    \item \textbf{Model options:} learned variance, EMA, progressive training.
\end{itemize}

% [Table 3 Placeholder: Ablations summary with mean metrics]
% [Figure 3 Placeholder: Guidance schedule sensitivity curves]

% --- Discussion and Limitations ---
\section{Discussion and Limitations}
\label{sec:limitations}
PKL guidance respects Poisson statistics and provides adaptive updates that strengthen corrections in low-count regions while avoiding over-smoothing in bright regions. The measured-forward-model integration yields robustness to realistic mismatches but remains sensitive to severe PSF errors and large misalignments. Computational cost is higher than single-pass CNNs, and our forward model assumes spatially invariant convolution.

% --- Conclusion ---
\section{Conclusion}
\label{sec:conclusion}
We presented PKL guidance for diffusion-based microscopy restoration on real grayscale data with measured PSFs and backgrounds. The method provides a principled alternative to L$_2$ guidance, with improved reconstruction quality and robustness. We include comprehensive ablations and will release code at submission.

\subsubsection*{Reproducibility Statement}
We will release code, configs, and evaluation scripts \emph{at submission} to reproduce the main tables and figures, including PSF loading and guidance implementations.

\bibliographystyle{iclr2026_conference}
\bibliography{bibliography}

\end{document}
